\documentclass{article} % For LaTeX2e
\usepackage{nips13submit_e,times}
\usepackage{hyperref}
\usepackage{url}
\usepackage{amsmath}
\usepackage{amsthm}
%\documentstyle[nips13submit_09,times,art10]{article} % For LaTeX 2.09


\title{A Data-Balanced Geohash for Distributed Spatiotemporal Database Indexing}


\author{
Taylor B.~Arnold \\
AT\&T Labs Research\\
33 Thomas Street\\
New York, NY 10007 \\
\texttt{taylor@research.att.com}
}

% The \author macro works with any number of authors. There are two commands
% used to separate the names and addresses of multiple authors: \And and \AND.
%
% Using \And between authors leaves it to \LaTeX{} to determine where to break
% the lines. Using \AND forces a linebreak at that point. So, if \LaTeX{}
% puts 3 of 4 authors names on the first line, and the last on the second
% line, try using \AND instead of \And before the third author name.

\newcommand{\fix}{\marginpar{FIX}}
\newcommand{\new}{\marginpar{NEW}}

\nipsfinalcopy % Uncomment for camera-ready version

\begin{document}

\maketitle

\begin{abstract}
We present a modification of the standard geohash algorithm
for which the data volume, rather than spatial area, is constant
for a given hash prefix length. This property is particularly
useful for indexing large distributed databases, where load
distribution of large range scans is an imporant aspect of
query performance. Distributed spatiotemporal databases, which
typically require interleaving spatial and temporal elements
into a single key, reap additional benefits from a balanced
geohash by creating a consistent balance between spatial and
temporal precision even across areas of varying data density.
We apply our algorithm to data generated proportional to population
as given by census block population counts provided from the
US Census Bureau. An efficent implementation for calculating an
arbitrary balanced geohash is also provided.
\end{abstract}

%%%%%%%%%%%%%%%%%%%%%%%%%%%%%%%%%%%%%%
\section{Introduction}

Queries over large distributed
databases often take the form of a series of large range scans;
balancin

%%%%%%%%%%%%%%%%%%%%%%%%%%%%%%%%%%%%%%
\section{A Balanced Geohash}

\subsection{Formal definition of standard geohash}

\subsection{Weighting scheme}

\subsection{Error analysis}

\begin{align}
K(x) &\equiv -\log_2 \left( | h(x) - f(x) | \right)
\end{align}

%%%%%%%%%%%%%%%%%%%%%%%%%%%%%%%%%%%%%%
\section{Application to Census Data}

\subsection{Population based}

\subsection{Robustness analysis}


%%%%%%%%%%%%%%%%%%%%%%%%%%%%%%%%%%%%%%
\section{Implementation}


%%%%%%%%%%%%%%%%%%%%%%%%%%%%%%%%%%%%%%
\section{Conclusions}




\end{document}
