\documentclass{article} % For LaTeX2e
\usepackage{nips13submit_e,times}
\usepackage{hyperref}
\usepackage{url}
\usepackage{amsmath}
\usepackage{amsthm}
\usepackage{amssymb}
%\documentstyle[nips13submit_09,times,art10]{article} % For LaTeX 2.09

\newtheorem{thm}{Theorem}

\title{An Entropy Maximizing Geohash for Distributed Spatiotemporal Database Indexing}


\author{
Taylor~B.~Arnold \\
AT\&T Labs Research\\
33 Thomas Street\\
New York, NY 10007 \\
\texttt{taylor@research.att.com}
}

% The \author macro works with any number of authors. There are two commands
% used to separate the names and addresses of multiple authors: \And and \AND.
%
% Using \And between authors leaves it to \LaTeX{} to determine where to break
% the lines. Using \AND forces a linebreak at that point. So, if \LaTeX{}
% puts 3 of 4 authors names on the first line, and the last on the second
% line, try using \AND instead of \And before the third author name.

\newcommand{\fix}{\marginpar{FIX}}
\newcommand{\new}{\marginpar{NEW}}

\nipsfinalcopy % Uncomment for camera-ready version

\begin{document}

\maketitle

\begin{abstract}
We present a modification of the standard geohash algorithm
for which the data volume, rather than spatial area, is constant
for a given hash prefix length. This property is particularly
useful for indexing large distributed databases, where load
distribution of large range scans is an imporant aspect of
query performance. Distributed spatiotemporal databases, which
typically require interleaving spatial and temporal elements
into a single key, reap additional benefits from a balanced
geohash by creating a consistent balance between spatial and
temporal precision even across areas of varying data density.
We apply our algorithm to data generated proportional to population
as given by census block population counts provided from the
US Census Bureau. An efficent implementation for calculating an
arbitrary balanced geohash is also provided.
\end{abstract}

%%%%%%%%%%%%%%%%%%%%%%%%%%%%%%%%%%%%%%
\section{Introduction}

Queries over large distributed
databases often take the form of a series of large range scans;
balancin

%%%%%%%%%%%%%%%%%%%%%%%%%%%%%%%%%%%%%%
\section{Entropy Balanced Geohash}

\subsection{A Formulation of the Standard Geohash Encoding}

A geohash is a scheme for mapping two-dimensional coordinates into a
hierarchical, one-dimensional encoding. It is explained in several
other sources, but we re-construct it here in a format which will be
most conducive to generalizations.

The first step is to map latitude and longitude coordinates in a
standard unit square; this is done by the following linear mapping:
\begin{align}
x &= \frac{\text{lon} + 180}{360} \\
y &= \frac{\text{lat} + 90}{180}
\end{align}
This choice is by convention, and any other method for mapping coordinantes
into the unit square is equally viable. The geohash formulation actually works
on any continuous two-dimensional data, and could be used  as-is to encode other
two-dimensional datasets.

The $x$ and $y$ coordinates need to be expressed in as a binary decimals.
Formally, we define the unique $x_i \in \{0,1 \}$ and $y_i \in \{0,1\}$
such that
\begin{align}
x &= \sum_{i=1}^{\infty} \frac{x_i}{2^i},\\
y &= \sum_{i=1}^{\infty} \frac{y_i}{2^i}.
\end{align}
A geohash representation of $(x,y)$ is constructed by interleaving these
binary digits. The $q$-bit geohash $g_q(\cdot,\cdot)$ can symbolically
be defined as
\begin{align}
g_q(x,y) &:=  \sum_{i=1}^{\lceil q/2 \rceil} \frac{x_i}{2^{2i-1}} +
              \sum_{i=1}^{\lfloor q/2 \rfloor} \frac{y_i}{2^{2i}}.
\end{align}
It is fairly easy to show that the geohash function is monotone increasing
in $q$, with the growth strictly bounded by $2^{-q}$, so that
\begin{align}
0 \leq g_{q+m}(x,y) - g_{q}(x,y) < \frac{1}{2^q}
\end{align}
For all $m$ greater than zero.

\subsection{Entropy}

A geohash is typically used as an index in the storing and querying of large
spatial processes. A simple theoretical model for a stream of spatial data can
be constructed by assuming that each observation is an independent identically
distributed random variable $\mathfrak{F}$ from some distribution over space.
Borrowing a concept from information theory, we can define the entropy of a geohash
over a given spatial distribution by the equation
\begin{align}
H(g_q) &:= -1 \sum_{v \in \mathcal{R}(g_q)} \mathbb{P} \left[ g_q(\mathfrak{F}) = v\right] \cdot
            \log_2 \left\{ \mathbb{P} \left[ g_q(\mathfrak{F}) = v\right] \right\} \label{entropyDef}
\end{align}
Where $\mathcal{R}(g_q)$ is the range of the $q$-bit geohash. It is a standard result
that the entropy of a discrete distribution is maximized by the uniform distribution.
Therefore we can use this as a proxy for how balanced a geohash is for a given distribution
of spatial data.

\subsection{The Generalized Geohash}

As the $q$-bit geohash function is bounded and monotonic in $q$, we can define the infinite
precision geohash, which we denote as simply $g(\cdot, \cdot)$, to be the limit
\begin{align}
\lim_{q \rightarrow \infty} g_q(x,y) &:= g(x,y).
\end{align}
With this continuous format, one can see that if we compose $g$ with an appropriate new function
$h$, the composition $h \circ g(x,y)$ which can be thought of as a rescaled version of the
traditional geohash. To be precise, we would like a function $h$ to have the following properties:
\begin{align}
&h: [0,1] \rightarrow [0,1], \label{hDef} \\
&h(0) = 0, \\
&h(1) = 1, \\
&x < y \iff h(x) < h(y). \label{monotoneEq}
\end{align}
Note that Equation~\ref{monotoneEq} implies that $h$ is also continuous. From here, we can define
the analogue to a $q$-bit geohash be truncating the binary representation of $w$, the value of
$h(z)$,
\begin{align}
h(z) &= \sum_{i=1}^{\infty} \frac{w_i}{2^i}
\end{align}
To the its first $q$-bits
\begin{align}
h_q(z) &:= \sum_{i=1}^{q} \frac{w_i}{2^i}.
\end{align}
In the remainder of this document, we refer the $h_q \circ g(x,y)$ as a generalized geohash.

\subsection{The Empirical Entropic Geohash}

We have introduced the concept of a generalized geohash in order to construct a spatial encoding
scheme which better optimizes the entropy as defined in Equation~\ref{entropyDef}.
Assume $\{z_i\}_{i=0}^N$ is a set of independent sample from realizations of the random variable
$\mathfrak{F}$. The empirical cumulative distribution function $G$ of the standard geohash function
$g(\cdot, \cdot)$ is given by
\begin{align}
G(t) &:= \frac{1}{N} \cdot \sum_{i=0}^{N} 1_{g(z_i) \leq t}, \, t \in [0,1]. \label{empGeohash}
\end{align}
From Equation~\ref{empGeohash} we can define the balanced entropy maximizing geohash function
$b$ (balanced), assuming that every point $z_i$ has a geohash $g(z_i)$ strictly between $0$ and
$1$, to be
\begin{align}
b^q(t) :=
  \begin{cases}
    0 &\mbox{if } t = 0\\
    1 &\mbox{if } t = 1\\
    \frac{N}{N+2} \cdot G^{-1}(t) &\mbox{if } \, \exists \, i \in \mathbb{Z} \, \text{s.t.} \, t = i / 2^{q} \\
    \text{linear interpolation of the above points} & \text{else}
  \end{cases} \label{balancedHash}
\end{align}
The balanced geohash is essentialy the inverse of the empirical function $G$, with some minor
variations to satisfy Equations~\ref{hDef}-\ref{monotoneEq}.
If the points $\{z_i\}$ are unique, and $N$ is sufficently large, the $q$-bit analogue $b_q$
of Equation~\ref{balancedHash} will have an entropy $H(b_q)$ of approximately equal to $q$.

More formally, we can prove the following bound on the entropy of the balanced geohash:
\begin{thm} \label{entropyThm}
Let $b_q^q$ be the entropy balanced geohash estimated from a sample of $N$ unique data points. Then,
with probability at least $1 - 2 e^{-0.49 \cdot 2^{-2q} N \cdot [1 - A]^2}$ the entropy
$H(b_q^q)$ is bounded by the following simultaneously for all values $A \in [0,1]$:
\begin{align}
H(b_q^q) &\geq q \cdot \frac{N}{N+2} \cdot A
\end{align}
\end{thm}
\begin{proof}
Let $F(\cdot)$ be the true cumulative distribution function of the variable $\mathfrak{F}$,
and $F_N(\cdot)$ be the empirical cumulative distribution function from a sample of $N$
independent observations. Setting $\epsilon = [1 - A] \cdot \frac{N}{N+2} \cdot 2^{-(q+1)}$,
the Dvoretzky–-Kiefer–-Wolfowitz inequality states that the follow holds for all values
$A \in [0,1]$ with probability $1 - e^{-2N\epsilon}$:
\begin{align}
|F(x) - F_n(x)| \leq [1 - A] \cdot \frac{N}{N+2} \cdot 2^{-(q+1)}
\end{align}
Therefore, the empirical entropy should be bounded as follows:
\begin{align}
H(b_q^q) &\geq -1 \cdot \sum_{i=0}^{2^q-1} \left[F(i/2^q) - F((i+1)/2^q)\right] \times
  \log_2 \left[F(i/2^q) - F((i+1)/2^q)\right] \\
&\geq -1 \cdot \sum_{i=0}^{2^q-1} \left[F_n(i/2^q) - F_n((i+1)/2^q) - 2\epsilon \right] \times
  \log_2 \left[F_n(i/2^q) - F_n((i+1)/2^q) - 2\epsilon \right] \\
&= -2^{q} \cdot \left[\frac{N}{N+2} \cdot \frac{1}{2^q} - 2 \epsilon \right] \times
  \log_2 \left[\frac{N}{N+2} \cdot \frac{1}{2^q} - 2 \epsilon  \right] \\
&= -1 \cdot \frac{N}{N+2} A \times \log_2 \left[\frac{N}{N+2} \cdot \frac{1}{2^q} \cdot A  \right] \\
&\geq q \cdot \frac{N}{N+2} A
\end{align}
Which, plugging in the appropriate $\epsilon$ into the probablity bound, yields the desired result.
\end{proof}
Plugging in $A=2/3$, the bound in Theorem~\ref{entropyThm} holds with probability greater than $0.99$
for a 5-bit balanced geohash when $N$ is at least $1e5$ and for a 10-bit geohash when $N$ is at
least $1e8$. We will see in the our empirical examples that the rate of convergence of the entropy
to $q$ is typically much faster.

%%%%%%%%%%%%%%%%%%%%%%%%%%%%%%%%%%%%%%
\section{Census Data Example}

\subsection{Population based}

\begin{table}[ht]
\centering
\begin{tabular}{rrrrrrrrrrrrrrrrrr}
  \hline
 & V2 & V3 & V4 & V7 & V8 & V9 & V10 & V11 & V12 & V13 & V14 & V15 & V16 & V17 & V18 & V19 & V20 \\
  \hline
alabama & 0.00 & 0.88 & 6.74 & 0.00 & 0.85 & 6.30 & 13.13 & 1.00 & 5.00 & 10.00 & 15.64 & 15.84 & 1.00 & 4.95 & 9.75 & 14.53 & 14.71 \\
  alaska & 0.41 & 1.94 & 5.41 & 0.43 & 1.94 & 4.91 & 10.36 & 1.00 & 5.00 & 9.91 & 12.26 & 12.45 & 1.00 & 4.89 & 9.57 & 11.64 & 11.79 \\
  arizona & 0.00 & 0.95 & 5.08 & 0.00 & 0.84 & 4.53 & 12.11 & 1.00 & 5.00 & 10.00 & 15.39 & 15.74 & 1.00 & 4.92 & 9.76 & 14.12 & 14.37 \\
  arkansas & 0.09 & 0.51 & 6.72 & 0.11 & 0.53 & 6.25 & 12.64 & 1.00 & 5.00 & 9.99 & 15.11 & 15.35 & 1.00 & 4.96 & 9.79 & 14.11 & 14.34 \\
  california & 0.00 & 0.78 & 6.59 & 0.00 & 0.78 & 6.23 & 13.88 & 1.00 & 5.00 & 10.00 & 16.91 & 17.65 & 0.99 & 4.95 & 9.85 & 16.14 & 16.74 \\
  colorado & 0.00 & 0.86 & 5.19 & 0.00 & 0.83 & 4.66 & 11.82 & 1.00 & 5.00 & 10.00 & 15.27 & 15.60 & 1.00 & 4.91 & 9.69 & 13.97 & 14.23 \\
  connecticut & 0.00 & 0.00 & 4.31 & 0.00 & 0.00 & 4.05 & 11.13 & 1.00 & 5.00 & 10.00 & 14.56 & 14.61 & 1.00 & 4.91 & 9.59 & 13.31 & 13.34 \\
  delaware & 0.00 & 0.98 & 3.55 & 0.00 & 0.94 & 3.20 & 9.51 & 1.00 & 5.00 & 9.94 & 12.89 & 12.92 & 0.97 & 4.86 & 9.34 & 11.51 & 11.53 \\
  D.C. & 0.00 & 0.00 & 1.22 & 0.00 & 0.00 & 1.23 & 7.24 & 1.00 & 5.00 & 9.76 & 11.33 & 11.33 & 0.99 & 4.94 & 9.44 & 10.48 & 10.48 \\
  florida & 0.00 & 0.98 & 6.57 & 0.00 & 0.97 & 6.25 & 13.72 & 1.00 & 5.00 & 10.00 & 16.46 & 16.99 & 1.00 & 4.97 & 9.83 & 15.51 & 15.94 \\
  georgia & 0.00 & 1.00 & 6.48 & 0.00 & 1.00 & 6.13 & 13.46 & 1.00 & 5.00 & 10.00 & 15.61 & 15.88 & 1.00 & 4.94 & 9.75 & 14.61 & 14.87 \\
  hawaii & 0.00 & 0.81 & 3.35 & 0.00 & 0.80 & 3.32 & 9.34 & 1.00 & 5.00 & 9.91 & 12.24 & 12.30 & 0.99 & 4.88 & 9.62 & 11.61 & 11.66 \\
  idaho & 0.74 & 1.41 & 5.44 & 0.77 & 1.40 & 5.19 & 11.30 & 1.00 & 5.00 & 9.99 & 14.35 & 14.57 & 1.00 & 4.93 & 9.69 & 13.41 & 13.57 \\
  illinois & 0.37 & 0.83 & 5.75 & 0.35 & 0.77 & 5.07 & 12.60 & 1.00 & 5.00 & 10.00 & 16.61 & 17.19 & 0.98 & 4.87 & 9.73 & 15.36 & 15.79 \\
  indiana & 0.00 & 0.75 & 6.52 & 0.00 & 0.73 & 6.14 & 12.83 & 1.00 & 5.00 & 10.00 & 15.99 & 16.39 & 1.00 & 4.94 & 9.74 & 14.73 & 15.06 \\
  iowa & 0.00 & 0.00 & 6.79 & 0.00 & 0.00 & 6.38 & 12.17 & 1.00 & 5.00 & 10.00 & 15.25 & 15.83 & 1.00 & 4.98 & 9.77 & 13.91 & 14.31 \\
  kansas & 0.00 & 0.38 & 6.01 & 0.00 & 0.33 & 5.62 & 11.76 & 1.00 & 5.00 & 10.00 & 15.00 & 15.59 & 1.00 & 4.95 & 9.72 & 13.68 & 14.12 \\
  kentucky & 0.00 & 0.00 & 6.55 & 0.00 & 0.00 & 6.08 & 12.71 & 1.00 & 5.00 & 10.00 & 15.09 & 15.27 & 1.00 & 4.95 & 9.77 & 14.17 & 14.34 \\
  louisiana & 0.32 & 0.32 & 6.08 & 0.28 & 0.28 & 5.49 & 12.35 & 1.00 & 5.00 & 10.00 & 15.29 & 15.54 & 1.00 & 4.93 & 9.75 & 14.31 & 14.51 \\
  maine & 0.49 & 0.59 & 5.89 & 0.46 & 0.56 & 5.46 & 11.39 & 1.00 & 5.00 & 9.99 & 13.95 & 14.03 & 1.00 & 4.89 & 9.64 & 13.01 & 13.09 \\
  maryland & 0.00 & 0.87 & 4.69 & 0.00 & 0.79 & 4.34 & 11.53 & 1.00 & 5.00 & 10.00 & 14.93 & 15.06 & 1.00 & 4.93 & 9.65 & 13.62 & 13.72 \\
  massachusetts & 0.00 & 0.00 & 4.72 & 0.00 & 0.00 & 4.32 & 11.58 & 1.00 & 5.00 & 10.00 & 15.54 & 15.68 & 0.98 & 4.91 & 9.71 & 14.45 & 14.55 \\
  michigan & 0.28 & 0.28 & 6.44 & 0.26 & 0.26 & 5.88 & 13.20 & 1.00 & 5.00 & 10.00 & 16.36 & 16.73 & 1.00 & 4.94 & 9.75 & 14.91 & 15.21 \\
  minnesota & 1.00 & 1.00 & 6.11 & 0.98 & 0.98 & 5.35 & 12.11 & 1.00 & 5.00 & 10.00 & 15.59 & 16.00 & 0.98 & 4.89 & 9.65 & 13.73 & 14.03 \\
  mississippi & 0.93 & 1.73 & 6.89 & 0.96 & 1.74 & 6.48 & 12.73 & 1.00 & 5.00 & 9.99 & 15.02 & 15.22 & 1.00 & 4.95 & 9.75 & 13.97 & 14.16 \\
  missouri & 0.23 & 0.61 & 6.52 & 0.24 & 0.60 & 6.08 & 12.89 & 1.00 & 5.00 & 10.00 & 15.84 & 16.26 & 1.00 & 4.92 & 9.76 & 14.70 & 15.09 \\
  montana & 0.03 & 0.98 & 6.33 & 0.04 & 1.00 & 5.79 & 11.13 & 1.00 & 5.00 & 9.99 & 14.09 & 14.38 & 1.00 & 4.93 & 9.69 & 13.18 & 13.40 \\
  nebraska & 0.00 & 0.32 & 5.60 & 0.00 & 0.32 & 5.21 & 11.16 & 1.00 & 5.00 & 10.00 & 14.78 & 15.28 & 1.00 & 4.93 & 9.66 & 13.39 & 13.71 \\
  nevada & 0.00 & 0.72 & 3.35 & 0.00 & 0.71 & 2.91 & 10.55 & 1.00 & 5.00 & 9.98 & 13.78 & 13.94 & 0.99 & 4.92 & 9.72 & 12.61 & 12.72 \\
  new\_hampshire & 0.01 & 0.01 & 4.70 & 0.01 & 0.01 & 4.48 & 10.70 & 1.00 & 5.00 & 9.98 & 13.65 & 13.70 & 1.00 & 4.86 & 9.48 & 12.47 & 12.51 \\
  new\_jersey & 0.00 & 0.17 & 4.69 & 0.00 & 0.17 & 4.15 & 11.56 & 1.00 & 5.00 & 10.00 & 15.83 & 16.01 & 0.96 & 4.87 & 9.68 & 14.36 & 14.49 \\
  new\_mexico & 0.00 & 0.84 & 5.46 & 0.00 & 0.83 & 4.89 & 11.23 & 1.00 & 5.00 & 9.99 & 14.43 & 14.70 & 0.98 & 4.90 & 9.68 & 13.29 & 13.46 \\
  new\_york & 0.00 & 0.28 & 5.28 & 0.00 & 0.26 & 4.15 & 11.76 & 1.00 & 5.00 & 10.00 & 16.25 & 16.64 & 0.97 & 4.83 & 9.74 & 14.99 & 15.25 \\
  north\_carolina & 0.00 & 0.89 & 6.89 & 0.00 & 0.90 & 6.59 & 14.01 & 1.00 & 5.00 & 10.00 & 16.08 & 16.32 & 1.00 & 4.97 & 9.82 & 15.21 & 15.44 \\
  north\_dakota & 0.00 & 0.78 & 5.71 & 0.00 & 0.73 & 4.61 & 9.80 & 1.00 & 5.00 & 9.98 & 13.71 & 14.07 & 0.99 & 4.88 & 9.45 & 11.86 & 12.02 \\
  ohio & 0.00 & 0.61 & 6.59 & 0.00 & 0.63 & 6.17 & 13.45 & 1.00 & 5.00 & 10.00 & 16.40 & 16.83 & 1.00 & 4.95 & 9.80 & 15.23 & 15.61 \\
  oklahoma & 0.00 & 0.05 & 6.25 & 0.00 & 0.05 & 5.67 & 12.35 & 1.00 & 5.00 & 10.00 & 15.35 & 15.80 & 1.00 & 4.95 & 9.76 & 14.19 & 14.57 \\
  oregon & 0.99 & 1.29 & 5.46 & 0.99 & 1.29 & 5.10 & 11.87 & 1.00 & 5.00 & 10.00 & 14.96 & 15.25 & 1.00 & 4.96 & 9.79 & 14.02 & 14.24 \\
  pennsylvania & 0.00 & 0.84 & 6.39 & 0.00 & 0.86 & 5.96 & 13.18 & 1.00 & 5.00 & 10.00 & 16.65 & 17.17 & 1.00 & 4.96 & 9.82 & 15.50 & 16.00 \\
  rhode\_island & 0.00 & 0.00 & 2.55 & 0.00 & 0.00 & 2.35 & 9.34 & 1.00 & 5.00 & 9.97 & 13.29 & 13.31 & 0.99 & 4.88 & 9.58 & 12.36 & 12.37 \\
  south\_carolina & 0.00 & 0.96 & 6.21 & 0.00 & 0.98 & 5.85 & 13.07 & 1.00 & 5.00 & 10.00 & 15.31 & 15.49 & 1.00 & 4.96 & 9.80 & 14.38 & 14.53 \\
  south\_dakota & 0.54 & 1.33 & 6.08 & 0.51 & 1.32 & 5.53 & 10.56 & 1.00 & 5.00 & 9.97 & 13.81 & 14.11 & 1.00 & 4.94 & 9.59 & 12.37 & 12.56 \\
  tennessee & 0.20 & 0.20 & 6.49 & 0.28 & 0.28 & 6.08 & 13.29 & 1.00 & 5.00 & 10.00 & 15.69 & 15.93 & 1.00 & 4.96 & 9.77 & 14.65 & 14.86 \\
  texas & 0.00 & 0.94 & 7.27 & 0.00 & 0.85 & 6.74 & 14.07 & 1.00 & 5.00 & 10.00 & 16.71 & 17.56 & 1.00 & 4.95 & 9.80 & 15.43 & 16.05 \\
  utah & 0.00 & 0.59 & 4.33 & 0.00 & 0.61 & 3.89 & 10.86 & 1.00 & 5.00 & 10.00 & 14.44 & 14.63 & 1.00 & 4.85 & 9.54 & 13.11 & 13.23 \\
  vermont & 0.04 & 0.04 & 5.32 & 0.04 & 0.04 & 4.98 & 10.58 & 1.00 & 5.00 & 9.93 & 12.92 & 12.95 & 0.99 & 4.92 & 9.57 & 12.01 & 12.04 \\
  virginia & 0.00 & 0.74 & 6.03 & 0.00 & 0.73 & 5.62 & 12.77 & 1.00 & 5.00 & 9.99 & 15.51 & 15.72 & 1.00 & 4.93 & 9.73 & 14.41 & 14.60 \\
  washington & 0.00 & 0.09 & 5.76 & 0.00 & 0.08 & 5.35 & 12.48 & 1.00 & 5.00 & 10.00 & 15.47 & 15.76 & 1.00 & 4.95 & 9.75 & 14.42 & 14.65 \\
  west\_virginia & 0.00 & 1.29 & 6.20 & 0.00 & 1.26 & 5.78 & 12.13 & 1.00 & 5.00 & 9.99 & 14.61 & 14.79 & 1.00 & 4.95 & 9.76 & 13.84 & 14.00 \\
  wisconsin & 1.08 & 1.08 & 6.59 & 0.93 & 0.93 & 5.95 & 12.49 & 1.00 & 5.00 & 10.00 & 15.88 & 16.25 & 1.00 & 4.94 & 9.77 & 14.37 & 14.65 \\
  wyoming & 0.00 & 0.00 & 5.75 & 0.00 & 0.00 & 5.42 & 10.15 & 1.00 & 5.00 & 9.97 & 13.43 & 13.59 & 1.00 & 4.92 & 9.59 & 12.42 & 12.52 \\
   \hline
\end{tabular}
\end{table}


%%%%%%%%%%%%%%%%%%%%%%%%%%%%%%%%%%%%%%
\section{Distributed Application}


%%%%%%%%%%%%%%%%%%%%%%%%%%%%%%%%%%%%%%
\section{Conclusions}




\end{document}
